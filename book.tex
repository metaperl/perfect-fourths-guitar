\documentclass[12pt]{report}
\usepackage{hyperref}   % use for hypertext links, including those to external documents and URLs
\usepackage{makeidx} 
\usepackage{gchords} 
\usepackage{rotating}
\usepackage{hyperref}

\makeindex

\begin{document}

\title{Perfect Fourths Guitar}
\author{Terrence Brannon}
\maketitle


\tableofcontents

\chapter{Intro}

This book covers an alternative tuning known as perfect fourths
tuning. Highly recommended companion texts are ``Total Scales,
Applications and Techniques'' and ``The 12 Notes of Music'', both by
Mark John Sternal.

All the names I know for perfect fourths guitar tuning are listed
below: 

\begin{enumerate}
\item EADGCF alternate tuning
\item p4 tuning
\item 4ths tuning
\item perfect fourths tuning
\item perfect forths tuning
\item quartal tuning
\end{enumerate}

There are plenty of articles motivating this style of playing. I think
\href{http://knol.google.com/k/guitar-tuning-in-fourths#}{this knol}
sums it up accurately.

\section{Acknowledgements}

\begin{enumerate}
\item Chris Koloian - my initial guitar teacher
\item Jamie Revell - an internet friend who plays in perfect fourths
  tuning 
\item The perfect fourths community on Facebook
  \href{http://www.facebook.com/#!/group.php?gid=183968224067}{(Guitar
    Tuning in 4ths)}

\end{enumerate}

\section{Heroes of p4 tuning}

Actve recording artists using p4 tuning:

\begin{enumerate}
\item Stanley Jordan
\item Allan Holdsworth
\item \href{http://www.youtube.com/user/tq105}{Tom Quayle}
\item \href{http://www.jamierevell.co.uk/}{Jamie Revell}
\item \href{http://justinperdue.com/}{Justin Perdue}
\item \href{http://www.youtube.com/user/mikibirta}{Miki Birta}
\end{enumerate}


\chapter{The Major Scale}

In this chapter, we will go over G Major Scale. Per Sternal, we will use 3
notes per string. What is a scale? A scale is a series of scale
positions. A scale position is a sequence of connected octave
shapes, allowing one to move across the fretboard. So, before we can
really get to studying scales, we have a few preliminary items. Let's
get going.

\section{Octave Shapes}

Octave shapes are a way to play a complete octave. They have a unique
start and end finger. It is important to note the start and end finger
of each octave shape, so that you can move forwards and backwards
across the fret board.

Instead of just reading about the shapes, be sure to play them. The
number just below the shape diagram indicates the suggest ``root
fret''. The reason for choosing that fret is so that you are playing
in the key of G --- the octave shapes work in all keys. It's just we
are starting with the key of G, so we play our octave shapes there for
now. 

\subsection{Shape 1}
\begin{sideways}
  \def\numfrets{10}
  \chord{t3}{f1bp{3}f3p{5}f5p{7},f1p{3}f3p{5}f5p{7},f1p{4}f2bp{5},n,n,n}{}
\end{sideways}

As you can see, shape 1 starts on finger 1 and ends on finger 2. It
covers 3 strings.

\subsection{Shape 2}
\begin{sideways}
  \def\numfrets{10}

  \chord{3}{n,n,f2bp{5}f4p{7},f1p{4}f2p{5}f4p{7},f1p{4}f3p{6}f4p{7},n}{}

\end{sideways}

As you can see, shape 2 starts on finger 2 and ends on finger 4. It
covers 3 strings.

\subsection{Shape 3}
\begin{sideways}

  \def\numfrets{16}
  \chord{8}{n,f4bp{11},f1p{8}f3p{10}f4p{11},f1p{8}f3p{10}f5p{12},f1bp{8},n}{}

\end{sideways}

As you can see, shape 3 starts on finger 4 and ends on finger 1. It
covers 4 strings. 

\section{Scale Positions}

There are 7 scale positions in any major scale. As we play increasingly
higher scale positions, we omit 1 or 2 strings from the octave shape
that opens the scale position, as shown in the following table:
\begin{table}[htbp]
\caption{Aspects of the Major Scale}
\begin{tabular}{|r|r|r|r|l|}
\hline
\multicolumn{1}{|l|}{Position} & \multicolumn{1}{l|}{Notes Omitted in initial octave shape} & \multicolumn{1}{l|}{Steps Omitted in First Octave Shape} & \multicolumn{1}{l|}{Strings til first root note} & Octave Shapes Played \\ \hline
1 & 0 & 0 & 3 & 1,2,3 \\ \hline
2 & 1 & 1 & 3 & 3,1,2 \\ \hline
3 & 2 & 1 & 2 & 2,3,1 \\ \hline
4 & 3 & 1 & 2 & 1,2,3 \\ \hline
5 & 4 & 2 & 2 & 3,1,2 \\ \hline
6 & 5 & 2 & 1 & 2,3,1 \\ \hline
7 & 6 & 2 & 1 & 1,2,3,1 \\ \hline
\end{tabular}
\label{}
\end{table}

Let's see what we can learn about the scale positions having looked at
the table. First, note how the octave shapes move through a regular
repeating cycle of 3, 1, 2 for the 1 and 2-string ommission scale
positions. What do we mean by string omisssion? It's simple. In the
prior section we looked at full octave shapes. When the initial octave
shape is played in scale positions 2-7, we omit 1 or 2 strings from
it. For example, in Position 2, we start with the 3rd octave shape,
but we omit the first string and only play the last 3 strings. 

\section{Let's Play!}

Ok, now that we've learned everything that a scale is, let's play.

In each position, be sure to pause or otherwise make mental note
whenever you play an root note.

\subsection{Position 1}

Position 1 is quite simple. Put your index finger on string 6 (the
thickest string, represented by the lowest line in the diagram). And
then follow the fingering. You will play all of shape 1, ending on
your second finger. That is your cue to continue with shape 2 (since
it starts on the second finger). Then we round out by playing as much
of shape 3 as we can before we run out of strings.

Then we go backwards. We hit our first root note on the 2nd string
(the 2nd thinnest string) on our fourth finger. Since our second
position \emph{ends} with the fourth finger, it's time to play that
backward until we end up on our second finger and encounter another
root note. At that point, we return to the start of position 1 by
playing shape 1 backwards.

\subsection {Position 2}

In Position 2 we start on the 5th fret. \footnote{You put your index
  finger where your 3rd finger was in the prior position}. Referencing
the table, we dont play the first string of shape 3, but start with
the second one. Then the same connected playing continues forward to
the 1st string and then backwards to the 6th string.

\subsection {You know the routine}

The same procedure of omitting strings from the initial shape in
positions 2-7 continues. And the same procedure of connecting shape to
shape backwards and forwards continues.

\section{Applications and Techniques}

Having taught you the scale, it is now time to crack open your copy of
MJS's ``Total Scales, Applications and Techniques'' and work through
all the applications and techniques with this scale per his
instructions in that book.

\subsection{What the Applications Reveal About the Scale} 

After you've worked through numerous applications, particularly the 2
and 3 string ones, you will start to see a single 7-string shape:


\def\numfrets{9}
\chord{t}{p{2}p{4}p{6},p{2}p{4}p{6},p{2}p{4}p{6},p{3}p{4}p{6},p{3}p{4}p{6},p{3}p{5}p{6},p{3}p{5}p{6}}{}


and then you will realize that each position simply starts this
7-string shape at a certain string! Here I enumerate the positions and
the string within the imaginary 7-string guitar that they each start on:

\begin{enumerate}
\item 2
\item 7
\item 5
\item 3
\item 1
\item 6
\item 4
\end{enumerate}


However, these numbers are not useful in and of themselves. Here is
something better to note:

\begin{quote}
At each position, you play 6 strings. When you move to the next
position, you repeat the 6th string and continue in order.
\end{quote}

Now, that is something practical. Let's see it in practice. So, let's
say you play the first position. As the enumeration just above shows,
we start on the second string of our imaginary 7-string guitar. So for
the first position, we play strings 2 through 7 on our imaginary
7-string. And because we ended on string 7, we continue with string 7
in the 2nd position and end on string 5. And so on.

In Sternal's book, Section 03 ``Bonus Speed and Memorization Drills''
can be mastered with the help of the tips in this section.

\end{document}
